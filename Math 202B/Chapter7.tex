
\newcommand{\RR}[0]{\mathcal{R}}
\newcommand{\PP}[0]{\mathcal{P}}
\newcommand{\mn}[0]{\mu\otimes\nu}

\chapter{Product Measures}
We now take some time to discuss construction of product measures. Let $(X,\Sm,\mu)$ and $(Y,\TT,\nu)$ be measure spaces. Obviously, one can theoretically define any measure on $X\times Y.$ However, throughout this chapter we will develop a natural measure $\mu\otimes\nu$ on the product space that reflects the measures on $X$ and $Y.$ We rely on a simple and intutive case: given a rectangle with side lengths $\ell$ and $w,$ we know that its measure (area) is the product $\ell w.$ \\

\noindent We will also discuss important theorems (Fubini and Tonelli) on integrals over the product of measure spaces, and indeed under certain conditions, the ``product'' integral is equal to the iterated integrals.

\section{Construction of the Product Measure}
\noindent For $E\in\Sm,$ $F\in\TT,$ we say that $E\times F$ is a \ub{measurable rectangle}. Let $\RR$ be the $\sigma$-algebra generated by the measurable rectangles, and let $\PP$ be the semi-ring of measurable rectangles (technically, for this we should consider half-open rectangles).

\begin{prop}
Define the function $\mn$ on $\PP$ by $\mn:=\mu(E)\;\nu(F).$ Then $\mn$ is a premeasure on $\PP.$ \\
\pf{
	It suffices to show countable additivity. \\
	Suppose $E\times F=\bigotimes_{n=1}^{\infty}(E_n\times F_n).$ Then $\chi_E(x)\;\chi_F(y)=\chi_{E\times F}(x,y)=\sum_{n=1}^{\infty}\chi_{E_n}(x)\chi_{F_n}(y).$ \\
	Fix $x\in X.$ Then 
	$$\sum_{n=1}^p\chi_{E_n}(x)\chi_{F_n}(y)\;\uparrow_p\;\sum_{n=1}^{\infty}\chi_{E_n}(x)\chi_{F_n}(y)=\chi_E(x)\chi_F(y),$$
	so by the Monotone Convergence Theorem, by integrating over $y,$ we get \\ $\chi_E(x)\nu(F)=\displaystyle\sum_{n=1}^{\infty}\chi_{E_n}(x)\nu(F_n).$ Now by the Monotone Convergence Theorem again and integrating over $x,$ we get the desired result.
}
\end{prop}

\noindent By Caratheodory's Theorem, we can simply restrict the outer measure $(\mn)^*$ to the measurable sets to obtain the product measure on the $\sigma$-algebra $\MM((\mn)^*),$ which turns out to be the completion of $\Sm\otimes\TT.$

\section{Iterated Integrals}
\noindent If $E\in\Sm,$ and $F\in\TT,$ it is clear that
$$\int\chi_G\;d(\mn)=(\mn)(G)=\mu(E)\nu(F)=\mu(E)\int\chi_F(y)\;d\nu(y)=$$
$$=\int\mu(E)\chi_F(y)\;d\nu(y)=\int\Big(\int\chi_E(x)\;d\mu(x)\Big)\chi_F(y)\;d\nu(y)=\int\Big(\int\chi_G(x,y)\;d\mu(x)\Big)d\nu(y).$$ \\

\noindent Our goal in this section is Fubini's Theorem. To prove this, it helps to discuss monotone classes.

\begin{defn}
A collection $\MM$ of subsets of $X$ is a \ub{monotone class} if whenever $A_n\in\MM$ and $A_n\uparrow A,$ then $A\in\MM,$ and also if $A_n\downarrow A,$ then $A\in\MM.$
\end{defn}

\noindent Given a collection $\mathcal{C}$ of subsets of $X,$ we let $\MM(\mathcal{C})$ denote the minimal monotone class containing $\mathcal{C},$ and we let $\sigma(\mathcal{C})$ denote the minimal such $\sigma$-ring.

\begin{frame*}
\noindent\ub{Monotone Class Theorem}: If $\RR$ is a ring, then $\MM(\RR)=\sigma(\RR).$ \\
\pf{
	Clearly $\sigma(\RR)\subseteq\MM(\RR).$ We must show that $\MM(\RR)\subseteq\sigma(\RR),$ and to do this it suffices to show that $\MM(\RR)$ is a ring. Indeed, assuming $\RR$ is a ring, if $E=\bigcup_{n=1}^{\infty}E_n,$ then take $F_n=\bigcup_{k=1}^n E_k$ so that $F_n\uparrow E.$ \\
	Let $E\in\MM(\RR),$ and let $L(E)=\{F\in\MM(\RR):\;E\setminus F,\;F\setminus E,\;E\cup F\in\MM(\RR)\}.$ \\
	We show that $L(E)$ is a monotone class:
	\begin{itemize}
	\item If $F_n\in L(E)$ and $F_n\uparrow F,$ then $E\setminus F_n\downarrow E\setminus F\in\MM(\RR),$ and $F_n\setminus E\uparrow F\setminus E\in\MM(\RR),$ and $F_n\cup E\uparrow F\cup E\in\MM(\RR),$ so that $F\in L(E).$ Similarly, if $F_n\downarrow F,$ then $F\in L(E).$ Thus $L(E)$ is a monotone class.
	\end{itemize}
	If $A\in\RR,$ then $L(A)$ is a monotone class, and $\RR\subseteq L(A),$ so $\MM(\RR)\subseteq L(A).$ Since \\ $L(A)\subseteq\MM(\RR),$ we therefore have $L(A)=\MM(\RR).$ But for every $E,F\in\MM(\RR),$ observe that $E\in L(F)$ iff $F\in L(E).$ Since $L(A)=\MM(\RR),$ every $E\in\MM(\RR)$ is in $L(A),$ and so $\RR\subseteq L(E),$ so also $L(E)=\MM(\RR),$ so $\MM(\RR)$ is a ring.
}
\end{frame*}

\noindent For any $G\subset X\times Y,$ let $G_x:=\{y\in Y:\;(x,y)\in G\}$ and $G^y:=\{x\in X:\;(x,y)\in G\}.$

\begin{thm}
Let $\mu$ and $\nu$ be $\sigma$-finite measures. For $G\in\Sm\otimes\TT,$ $x\in X,$ $y\in Y,$ we have $G_x\in\TT$ and $G^y\in\Sm.$ The maps $x\mapsto\nu(G_x)$ and $y\mapsto\mu(G^y)$ are $\Sm$- and $\TT$-measurable, respectively, and 
$$(\mn)(G)=\int\nu(G_x)\;d\mu(x)=\int\mu(G^y)\;d\nu(y).$$
\pf{
	Note that the results hold for $G$ in the ring generated by $\{E\times F:\;E\in\Sm,\;F\in\TT\}.$ Let $\mathcal{C}$ denote the set of all subsets for which the theorem holds. Then $\mathcal{C}$ is nonempty, and we will show that $\mathcal{C}$ is a monotone class. \\
	If $G_n\in\mathcal{C},$ and $G_n\uparrow G,$ then for $x\in X,$ $y\in Y,$ we have $(G_n)_x\in\TT,$ $(G_n)^y\in\Sm,$ $(G_n)_x\uparrow G_x,$ and $(G_n)^y\uparrow G^y.$ \\
	Then $\nu((G_n)_x)\uparrow\nu(G_x),$ so $x\mapsto\nu(G_x)$ is $\Sm$-measurable. Similarly, $y\mapsto\mu(G^y)$ is $\TT$-measurable. By the Monotone Convergence Theorem, 
	$$\int\nu((G_n)_x)\;d\mu(x)\uparrow\int\nu(G_x)\;d\mu(x).$$
	We also have 
	$$(\mn)(G_n)=\int\nu((G_n)_x)\;d\mu(x)=\int\mu((G_n)^y)\;d\nu(y)$$
	so that $(\mn)(G_n)\uparrow(\mn)(G).$ To show the decreasing case, we need $\sigma$-finiteness. Assume $\mu(X)<\infty,$ $\nu(Y)<\infty,$ $G_n\downarrow G$ with $G_n\in\mathcal{C}.$ Then $(G_n)_x\downarrow G_x$ so $G_x\in\TT$ and $(G_n)^y\downarrow G^y,$ so $G^y\in\Sm.$ As done above, we get that $x\mapsto\nu(G_x)$ is $\Sm$-measurable, and $y\mapsto\mu(G^y)$ is $\TT$-measurable. \\
	Since $\mu(X)<\infty,$ we have
	$$\int\nu((G_n)_x)\;d\mu(x)\downarrow\int\nu(G_x)\;d\mu(x)\implies (\mn)(G_n)\downarrow (\mn)(G).$$
	In general, $X=\bigoplus_{n=1}^{\infty} X_n$ and $Y=\bigoplus_{n=1}^{\infty}Y_n$ where $\mu(X_n)<\infty,$ $\nu(Y_n)<\infty.$ By applying the result to each $X_n$ and $Y_n,$ and applying additivity, we get the desired result.
}
\end{thm}

\begin{prop}
Let $f$ be an integrable simple function on $X\times Y.$ Then
$$\int_{X\times Y}f\;d(\mn)=\int\Big(\int f(x,y)\;d\mu(x)\Big)\;d\nu(y)=\int\Big(\int f(x,y)\;d\nu(y)\Big)d\mu(x).$$
\end{prop}

\noindent We need to know when $f\in\mathcal{L}^1.$ It suffices to have $f$ measurable and the map \\ $(x,y)\mapsto\norm{f(x,y)}$ is $(\mn)$-integrable.

\begin{prop}
Let $\mu,\nu$ be $\sigma$-finite. Let $f$ on $X\times Y$ be $\Sm\otimes\TT$-measurable and $f\geq 0.$ Then
$$\int f\;d(\mn)=\iint f(x,y)\;d\mu(x)\;d\nu(y)=\iint f(x,y)\;d\nu(y)\;d\mu(x).$$
\pf{
	Let $\{f_n\}$ be an increasing sequence of integrable simple functions, with $f_n\uparrow f$ pointwise. By the Monotone Convergence Theorem, since $(f_n)^y\uparrow f^y,$ we have
	$$\int (f_n)^y\;d\mu\;\uparrow\;\int f^y\;d\mu\implies\int\Big(\int (f_n)^y\;d\mu\Big)d\nu\;\uparrow\;\int\Big(\int f^y\;d\mu\Big)d\nu.$$
	But $\displaystyle\int\Big(\int (f_n)^y\;d\mu\Big)d\nu=\int f_n\;d(\mn)\;\uparrow\;\int f\;d(\mn).$
}
\end{prop}

\begin{cor}
If $f^y$ is $\mu$-integrable for a.e. $y,$ and $f\geq 0,$ and if $y\mapsto\int f^y\;d\mu$ is $\nu$-integrable, then $f\in\mathcal{L}^1(X\times Y,\Sm\otimes\TT,\mn,\mathbb{R}).$
\end{cor}

\begin{frame*}
\noindent \ub{Tonelli's Theorem}: Let $f$ be a $B$-valued $\Sm\otimes\TT$-measurable function. Let \\ $g(x,y)=\norm{f(x,y)}_B.$ If $g^y$ is $\mu$-integrable for a.e. $y,$ and if $y\mapsto\int g^y\;d\mu$ is $\nu$-integrable, then $f\in\mathcal{L}^1(X\times Y,\Sm\otimes\TT,\mn,\mathbb{R}).$ \\
\pf{
	By the corollary above, $g\in\mathcal{L}^1,$ thus $f\in\mathcal{L}^1.$
}
\end{frame*}

\begin{frame*}
\noindent\ub{Fubini's Theorem}: If $f\in\mathcal{L}^1(X\times Y,\Sm\otimes\TT,\mn,B),$ then $f^y$ is $\mu$-integrable for a.e. $y,$ and $y\mapsto\int f^y\;d\mu$ is $\nu$-integrable, and
$$\int f\;d(\mn)=\int\Big(\int f^y\;d\mu\Big)d\nu=\int\Big(\int f_x\;d\nu\Big)d\mu.$$
\pf{
	Let $g(x,y)=\norm{f(x,y)}_B,$ integrable. Let $\{f_n\}$ be integrable simple functions such that $f_n\rightarrow f$ pointwise. Then $\norm{f_n(x,y)}\leq 2g(x,y).$ \\
	By the Dominated Convergence Theorem,
	$$\int f\;d(\mn)=\lim_{n\rightarrow\infty}\int \underbrace{f_n}_{\text{ISF}}\;d(\mn)=\lim_{n\rightarrow\infty}\int\Big(\int f_n^y\;d\mu\Big)d\nu.$$
	Now $\displaystyle\int (f_n)^y\;d\mu$ for all $y$ such that $g^y$ is integrable is dominated by $\displaystyle\int g^y\;d\mu.$ Therefore $\displaystyle\int f\;d(\mn)=\int\Big(\int f^y\;d\mu\Big)d\nu.$
}
\end{frame*}

\noindent *Remark: The assumption that we are working with $\sigma$-finite measures is important. Here is an example where the iterated integrals and the integral with respect to the product measure are all not equal to each other: \\

\noindent Let $X=Y=[0,1],$ with $\Sm=\TT=$ the Borel $\sigma$-algebra on $[0,1].$ Let $\mu$ be Lebesgue measure, but let $\nu$ be counting measure (hence $\nu$ is not $\sigma$-finite). Let $D=\{(x,x):\;x\in[0,1]\}$ (the diagonal of $X\times Y$). Then one can check that
\begin{itemize}
\item $\displaystyle\iint\chi_D\;d\mu\;d\nu=0,$
\item $\displaystyle\iint\chi_D\;d\nu\;d\mu=1,$
\item $\displaystyle\int\chi_D\;d(\mn)=\infty.$
\end{itemize}
















