\chapter{Measure Theory}

\section{Fundamentals}

\begin{defn}
Let $X$ be a set, and let $\MM$ be a collection of subsets of $X.$ We say $\MM$ is a \ub{ring} if 
\begin{enumerate}
\item[(1)] $E\cup F\in\MM$ for any $E,F\in\MM.$
\item[(2)] $E\setminus F\in\MM$ for any $E,F\in\MM.$
\end{enumerate}
$\MM$ is an \ub{algebra/field} if $\MM$ is a ring, and $X\in\MM.$
\end{defn}

\noindent Note that (1) and (2) imply that if $E,F\in\MM,$ then $E\cap F\in\MM.$ \\ \\
\Ex Clearly, $\mathcal{P}(X)$ is a ring and an algebra on $X.$ \\
\Ex The set $\MM$ of all finite subsets of $X$ is a ring, but it is not an algebra if $X$ is infinite. \\
\Ex If $X=\R{},$ then $\MM:=\{[a,b):\;a<b\}$ is a ring.

\begin{defn}
$\FF$ is a \ub{$\sigma$-ring} on $X$ if
\begin{enumerate}
\item[(1)] $\FF$ is a ring
\item[(2)] If $\{E_n\}_{n=1}^{\infty}$ is a countable collection of elements in $\FF,$ then $\bigcup\limits_{n=1}^{\infty} E_n\in\FF.$
\end{enumerate}
$\FF$ is a \ub{$\sigma$-algebra/field} if it is a $\sigma$-ring, and $X\in\FF.$
\end{defn}

\noindent Note that $\sigma$-rings are also closed under countable intersections.

\begin{prop} Let $X$ be a set, and let $\{\FF_{\alpha}\}$ be a collection of rings/algebras/$\sigma$-rings/$\sigma$-algebras on $X.$ Then $\bigcap_{\alpha}\FF_{\alpha}$ is also a ring/algebra/$\sigma$-ring/$\sigma$-algebra, respectively. \\ \\
\pf{\textit{Exercise}.}
\end{prop}

\begin{prop}
For any family of sets $\mathcal{A}\se\mathcal{P}(X),$ there is a smallest ring/algebra/$\sigma$-ring/$\sigma$-algebra that contains $\mathcal{A}.$ \\ \\
\pf{The smallest such ring/algebra/$\sigma$-ring/$\sigma$-algebra is in fact the intersection of all rings/algebras/$\sigma$-rings/$\sigma$-algebras that contain $\mathcal{A}.$}
\end{prop}

\noindent The smallest such ring/algebra/$\sigma$-ring/$\sigma$-algebra containing $\mathcal{A}$ is the ring/algebra/$\sigma$-ring/$\sigma$-algebra \ub{generated} by $\mathcal{A}.$ \\ The smallest $\sigma$-ring generated by $\mathcal{A}$ will be denoted by $\Sm(\mathcal{A}).$

\begin{defn}
If $(X,\TT)$ is a topological space, then the $\sigma$-algebra generated by $\TT$ is called the \ub{Borel $\sigma$-algebra}. \\
If $(X,\TT)$ is a locally compact space, let $\CC$ denote the collection of all compact sets. The $\sigma$-ring generated by $\CC$ is called the \ub{Borel $\sigma$-ring}.
\end{defn}

\noindent\Ex Let $X$ be an uncountable set with the discrete topology. Then compact sets are finite sets, so the Borel $\sigma$-ring consists of all countable subsets of $X.$

\begin{defn}
Let $\mathcal{A}$ be a collection of subsets of a set $X.$ A function $\mu:\mathcal{A}\rightarrow [0,\infty]$ is \ub{additive} if for all disjoint $E,F\in\mathcal{A}$ such that $E\cup F\in\mathcal{A},$ we have $\mu(E\cup F)=\mu(E)+\mu(F).$ \\
$\mu$ is \ub{countable additive} if whenever $\{E_n\}_{n=1}^{\infty}$ is a countable collection of mutually disjoint sets such that $\bigcup\limits_{n=1}^{\infty} E_n\in\mathcal{A},$ we have $\mu\Big(\bigcup_{n=1}^{\infty} E_n\Big)=\sum\limits_{n=1}^{\infty}\mu(E_n).$
\end{defn}

\begin{defn}
A \ub{measure} on $X$ consists of a $\sigma$-ring $\Sm$ on $X,$ together with a function $\mu:\Sm\rightarrow[0,\infty]$ that is countably additive.
\end{defn}

\noindent\Ex Let $X$ be a set with the $\sigma$-field $\mathcal{P}(X).$ Then $\mu(E)=|E|,$ where $|E|=\infty$ if $E$ is infinite, is a measure (we could also have the $\sigma$-field be all countable subsets). This measure is called \textit{counting measure}.

\section{Borel Measures on $\R{}$}
We will develop Borel measures in more generality. That is, let $\alpha:\R{}\rightarrow\R{}$ be any function that is non-decreasing and left continuous, i.e., for any $t,$ we have $\lim\limits_{\epsilon\downarrow 0}\;\alpha(t-\epsilon)=\alpha(t).$ If we weren't general, we would have used $\alpha(t)=t$. These will lead to the ``usual'' measure on $\R{}.$ \\ \\
Let $P=\{[a,b):a<b\in\R{}\},$ and define $\mu$ on $P$ by $\mu([a,b))=\alpha(b)-\alpha(a).$ We will show that $\mu$ is countably additive.

\begin{defn}
Let $X$ be a set, $P$ a collection of subsets. Then $P$ is a \ub{pre-ring/semi-ring} if
\begin{enumerate}
\item[(1)] $E,F\in P$ implies $E\cap F\in P.$
\item[(2)] $E,F\in P$ implies $E\setminus F$ is a finite disjoint union of elements in $P.$
\end{enumerate}
\end{defn}

\noindent Then it is easy to see that $P=\{[a,b):a<b\}$ is a pre-ring on $\R{}.$

\begin{defn}
For a pre-ring $P,$ a function $\mu:P\rightarrow[0,\infty]$ is a \ub{premeasure} on $P$ if it is countably additive.
\end{defn}

\begin{thm}
The function $\mu$ on $P$ defined above is countably additive. That is, $\mu$ is a pre-measure. \\ \\
\pf{
	Let $\{[a_j,b_j)\}_{j=1}^{\infty}$ be disjoint, and let $[a,b)=\bigcup_{j=1}^{\infty}[a_j,b_j)$ be their union. \\
	We will first show that $\sum_{j=1}^{\infty}\mu([a_j,b_j))\leq\mu([a,b)).$ To show this, let $n\in\mathbb{N}.$ Then $\sum_{j=1}^n\mu([a_j,b_j))=\sum_{j=1}^n[\alpha(b_j)-\alpha(a_j)].$ Without loss of generality, suppose $a_1<a_2<\cdots<a_n.$ Since the intervals are disjoint, $b_j\leq a_{j+1}$ for all $j,$ so $\alpha(b_j)\leq\alpha(a_{j+1}).$ Then
	$$\sum_{j=1}^n\mu([a_j,b_j))\leq [\alpha(b_n)-\alpha(a_n)]+[\alpha(a_n)-\alpha(a_{n-1})]+\cdots+[\alpha(a_2)-\alpha(a_1)]\leq \alpha(b)-\alpha(a).$$
	Since this holds for any $n,$ it holds for the limit. \\ \\
	Conversely, let $\epsilon>0.$ Let $\{\epsilon_j\}$ be a sequence such that $\sum_{j=1}^{\infty}\epsilon_j<\epsilon/2,$ and let $b'$ be a number such that $\alpha(b')+\epsilon/2\geq\alpha(b).$ For each $j,$ choose $a_j'<a_j$ such that $\alpha(a_j')+\epsilon_j\geq\alpha(a_j).$ \\
	Then $[a,b']\se[a,b)=\bigcup_{j=1}^{\infty}[a_j,b_j)\se\bigcup_{j=1}^{\infty}(a_j',b_j).$ Since $[a,b']$ is compact, and $\bigcup_{j=1}^{\infty}(a_j',b_j)$ is an open cover, there must be a finite subcover. Relabel these intervals: $(a_1',b_1),\hdots,(a_n',b_n),$ such that $a_0\in(a_1',b_1),$ and $b_j\in(a_{j+1}',b_{j+1}),$ so $a_{j+1}'<b_j,$ and also $b'\leq b_n.$ Then
	$$\sum_{j=1}^{\infty}[\alpha(b_j)-\alpha(a_j')]=\alpha(b_n)+\alpha(b_{n-1})-\alpha(a_n'),$$
	where $\alpha(b_{n-1})-\alpha(a_n')\geq 0.$ Then
	$$\alpha(b)-\alpha(a)\;\leq\;\alpha(b')+\frac{\epsilon}{2}-\alpha(a)\;\leq\;\alpha(b_n)+\frac{\epsilon}{2}-\alpha(a)\;\leq\;\alpha(b_n)-\alpha(a_1)+\frac{\epsilon}{2}\;\leq$$
	$$\leq \alpha(b_n)+[\alpha(b_{n-1})-\alpha(a_n')]+[\alpha(b_{n-2})-\alpha(a_{n-1}')]+\cdots+\alpha(a_1')+\frac{\epsilon}{2}\;=$$
	$$=\sum_{j=1}^n[\alpha(b_j)-\alpha(a_j')]\;\leq\;\sum_{j=1}^n[\alpha(b_j)-\alpha(a_j)+\epsilon_j]+\frac{\epsilon}{2}\;\leq\;\sum_{j=1}^{\infty}\mu([a_j,b_j))+\epsilon.$$
	Therefore, $\mu([a,b))\leq\sum\limits_{j=1}^{\infty}\mu([a_j,b_j)).$
}
\end{thm}

\noindent Note that if $P$ is a pre-ring, then for $E,F_1,F_2\in P,$ there exist $\{G_{jk}\}_{j,k=1}^{n_1,n_2}\se P,$ all disjoint, such that 
$$(E\setminus F_1)\setminus F_2=\Big(\bigcup_{j=1}^{n_1} G_j\Big)\setminus F_2=\bigcup_{j=1}^{n_1}\bigcup_{k=1}^{n_2} G_{jk}.$$
Then for $F_1,\hdots,F_{\ell}\in P,$ we have $E\setminus \bigcup_{i=1}^{\ell} F_i=\bigcup_{j=1}^p G_j,$ where $G_j\in P.$

\begin{prop}
If $\mu:P\rightarrow[0,\infty],$ and $\mu$ is finitely additive, and if $E\se\bigcup_{j=1}^n F_j,$ where the $F_j$'s are disjoint, and $E,F_j\in P,$ then $\mu(E)\leq\sum_{j=1}^n\mu(F_j).$ \\ \\
\pf{$$\bigcup_{j=1}^n F_j = E\cup\Big(\bigcup_{j=1}^n F_j\setminus E\Big)=E\cup\Big(\bigcup_{j=1}^{n_1}\bigcup_{k=1}^{n_2} G_{jk}\Big),$$
	for some $\{G_{jk}\}\se P$ disjoint. Furthermore, $E$ and $\bigcup_{j=1}^{n_1}\bigcup_{k=1}^{n_2} G_{jk}$ are disjoint, so
	$$\sum_{j=1}^n\mu(F_j)=\mu(E)+\sum_{j=1}^{n_1}\sum_{k=1}^{n_2}\mu(G_{jk})\geq\mu(E).$$
}
\end{prop}

\begin{cor}
If $E\se F,$ then $\mu(E)\leq\mu(F).$
\end{cor}

\begin{defn}
A function $\mu$ satisfying $\mu(E)\leq\mu(F)$ whenever $E\se F$ is called \ub{monotone}. \\ \\
For a family of sets $\FF,$ we say $\mu:\FF\rightarrow\R{}$ is \ub{countably subadditive} if whenever $E\se\bigcup_{j=1}^{\infty} F_j,$ where $E,F_j\in\FF,$ then $\mu(E)\leq\sum_{j=1}^{\infty}\mu(F_j).$
\end{defn}

\begin{prop}
If $(P,\mu)$ is a pre-measure, then $\mu$ is countably subadditive. \\ \\
\pf{
	Suppose $E\se\bigcup_{j=1}^{\infty} F_j.$ Then $E=E\cap\Big(\bigcup_{j=1}^{\infty} F_j\Big)=\bigcup_{j=1}^{\infty}(E\cap F_j).$ \\
	Note that $\mu(E\cap F_j)\leq\mu(F_j),$ so it suffices to show the result for $E=\bigcup_{j=1}^{\infty} F_j.$ \\
	Define $H_1=F_1,$ and for $n>1,$ define $H_n=F_n\setminus\bigcup_{j=1}^{n-1}F_j.$ Then clearly $H_1,H_2,\hdots$ are all disjoint, and clearly $\bigcup_{j=1}^{\infty} H_j=\bigcup_{j=1}^{\infty} F_j.$ Now for each $j,$ there are sets $\{G_{jk}\}_{k=1}^{n_j}$ such that $H_j=\bigcup_{k=1}^{n_j}G_{jk},$ so
	$$\mu(E)=\mu\Big(\bigcup_{j=1}^{\infty} H_j\Big)=\sum_{j=1}^{\infty}\mu(H_j)=\sum_{j=1}^{\infty}\mu\Big(F_j\setminus\bigcup_{k=1}^{j-1}F_k\Big)=$$
	$$=\sum_{j=1}^{\infty}\mu\Big(\sum_{k=1}^{n_j} G_{jk}\Big)=\sum_{j=1}^{\infty}\sum_{k=1}^{n_j}\mu(G_{jk})\leq\sum_{j=1}^{\infty}\mu(F_j).$$
}
\end{prop}

\section{Outer Measures}
\begin{defn}
Let $\FF$ be a family of subsets of $X.$ Let $A\se X.$ Then $A$ is \ub{countably covered} by $\FF$ if there is a sequence $\{F_j\}$ of subsets of $\FF$ with $A\se\bigcup_{j=1}^{\infty} F_j.$
\end{defn}

\noindent Let $\HH(\FF)$ denote the collection of all subsets of $X$ that are countably covered by $\FF.$ \\
Properties of $\HH(\FF)$:
\begin{enumerate}
\item[(i)] $\HH(\FF)$ is a $\sigma$-ring
\item[(ii)] $\HH(\FF)$ is \ub{hereditary}, i.e., if $A\in\HH(\FF),$ and $B\se A,$ then $B\in\HH(\FF).$
\end{enumerate}

\noindent Given $\HH(\FF),$ let $\mu:\FF\rightarrow[0,\infty]$ be any function. Define $\mu^*:\HH(\FF)\rightarrow[0,\infty]$ by
$$\mu^*(A)=\inf\Big\{\sum_{j=1}^{\infty}\mu(F_j):\;A\se\bigcup_{j=1}^{\infty} F_j,\; F_j\in\FF\Big\}.$$

\begin{defn}
Let $\HH$ be a hereditary $\sigma$-ring of subsets on $X.$ A function $\nu:\HH\rightarrow[0,\infty]$ is an \ub{outer measure} if
\begin{enumerate}
\item[(i)] $\nu(\varnothing)=0$
\item[(ii)] $\nu$ is monotone
\item[(iii)] $\nu$ is countably subadditive
\end{enumerate}
\end{defn}

\begin{prop}
$\mu^*:\HH(\FF)\rightarrow[0,\infty]$ defined as above is an outer measure. \\ \\
\pf{
	Clearly, $\mu^*(\varnothing)=0,$ and monotonicity is trivial. \\
	Suppose $A\se\bigcup_{j=1}^{\infty}B_j.$ Let $\epsilon>0,$ and suppose $\{\epsilon_j\}_{j=1}^{\infty}$ satisfies $\sum_{j=1}^{\infty}\epsilon_j<\epsilon.$ \\
	For each $j,$ there is a collection $\{B_{jk}\}_{k=1}^{\infty}\se\FF$ such that $B_j\se\bigcup_{k=1}^{\infty} B_{jk}$ and $\mu^*(B_j)+\epsilon_j>\sum\limits_{k=1}^{\infty}\mu(B_{jk}).$ Then 
	$$\mu^*(A)\leq\mu^*\Big(\bigcup_{j=1}^{\infty} B_j\Big)\leq\sum_{j=1}^{\infty}\sum_{k=1}^{\infty}\mu(B_{jk})<\sum_{j=1}^{\infty}\Big(\mu^*(B_j)+\epsilon_j\Big)<\sum_{j=1}^{\infty}\mu^*(B_j)+\epsilon.$$
	It follows that $\mu^*(A)\leq\sum_{j=1}^{\infty}\mu^*(B_j),$ so $\mu^*$ is countably subadditive.
}
\end{prop}

\begin{prop}
Let $(P,\mu)$ be a pre-measure. Then $\mu^*$ agrees with $\mu$ on $P.$ \\ \\
\pf{Let $E\in P.$ Clearly, $\mu^*(E)\leq\mu(E).$ However, if $E\se\bigcup_{j=1}^{\infty} E_j,$ then $\mu(E)\leq\sum_{j=1}^{\infty}\mu(E_j),$ so $\mu^*(E)\geq\mu(E),$ hence $\mu^*(E)=\mu(E).$}
\end{prop}

\section{From Outer Measure to Measure}
\begin{defn}
Let $\nu$ be an outer measure on a hereditary $\sigma$-ring $\HH.$ Then $E\in\HH$ is \ub{$\nu$-measurable} if for any $A\in\HH,$
$$\nu(A)=\nu(A\cap E)+\nu(A\cap E^c).$$
\end{defn}

\noindent Note that by subadditivity, we always have $\nu(A)\leq\nu(A\cap E)+\nu(A\cap E^c).$ \\ \\
From now on, let $\oplus$ denote a disjoint union. \\
Let $\MM(\nu)$ denote the set of all $\nu$-measurable sets in $\HH.$

\begin{frame*}
\noindent\ub{Caratheodory's Theorem} $\MM(\nu)$ is a $\sigma$-ring, and $\nu$ restricted to $\MM(\nu)$ is a measure. \\ \\
\pf{First, we show that $\MM(\nu)$ is a ring. \\
	Let $E,F\in\MM(\nu).$ Let $A\in\HH.$ Then
	$$\nu(A\cap(E\cup F))+\nu(A\cap(E\cup F)^c)=$$
	$$=\nu\Big((A\cap E)\oplus (A\cap E^c\cap F)\Big)+\nu\Big((A\cap E^c)\cap F^c\Big)\leq $$
	$$\leq \nu(A\cap E)+\nu((A\cap E^c)\cap F)+\nu((A\cap E^c)\cap F^c)=\nu(A\cap E)+\nu(A\cap E^c)=\nu(A),$$
	so $E\cup F\in\MM(\nu).$ On the other hand,
	$$\nu\Big(A\cap(E\cap F^c)\Big)+\nu\Big(A\cap(E^c\cup F)\Big)=$$
	$$=\nu\Big((A\cap E)\cap F^c\Big)+\nu\Big((A\cap E^c)\oplus(A\cap E\cap F)\Big)\leq$$
	$$\leq\nu(A\cap E\cap F^c)+\nu(A\cap E^c)+\nu(A\cap E\cap F)=\nu(A\cap E)+\nu(A\cap E^c)=\nu(A),$$
	so $E\setminus F\in\MM(\nu).$ Therefore $\MM(\nu)$ is a ring.\\ \\
	We now show that $\nu$ is finitely additive on $\MM(\nu).$\\
	Let $E,F\in\MM(\nu)$ be disjoint, and let $A=E\cup F.$ Then $E$ ``splits'' $A$:
	$$\nu(A)=\nu(A\cap E)+\nu(A\cap E^c)=\nu(E)+\nu(F).$$
	Furthermore, for $E,F$ disjoint and for any $A\in\HH,$
	$$\nu\Big((A\cap E)\cup(A\cap F)\Big)=\nu(A\cap E)+\nu(A\cap F).$$ 
	Finally, we show that $\MM(\nu)$ is a $\sigma$-ring. \\
	Let $\{E_j\}_{j=1}^{\infty}\se\MM(\nu).$ Since $\MM(\nu)$ is a ring, we can construct, as done previously, disjoint sets $\{F_j\}_{j=1}^{\infty}$ such that $\bigcup_{j=1}^{\infty} E_j=\bigcup_{j=1}^{\infty} F_j,$ where $F_j\in\MM(\nu).$ Let $A\in\HH.$ For any $m\geq 1,$ we know that
	$$\nu(A)=\nu\Big(A\cap\Big(\bigcup_{j=1}^m F_j\Big)\Big)+\nu\Big(A\cap\Big(\bigcup_{j=1}^m F_j\Big)^c\Big)\geq\sum_{j=1}^m\nu(A\cap F_j)+\nu\Big(A\cap\Big(\bigcup_{j=1}^m F_j\Big)^c\Big)\geq$$
	$$\geq\nu\Big(\bigcup_{j=1}^m(A\cap F_j)\Big)+\nu\Big(A\cap\Big(\bigcup_{j=1}^m F_j\Big)^c\Big)=\nu\Big(A\cap\Big(\bigcup_{j=1}^m F_j\Big)\Big)+\nu\Big(A\cap\Big(\bigcup_{j=1}^m F_j\Big)^c\Big).$$
	In particular, this holds for $A=\bigcup_{j=1}^{\infty} F_j,$ so
	$$\nu(A)=\sum_{j=1}^{\infty}\nu(A\cap F_j)=\sum_{j=1}^{\infty}\nu(F_j),$$
	so $\nu$ restricted to its $\nu$-measurable sets $\MM(\nu)$ is a measure.
}
\end{frame*}

\begin{prop}
Let $(P,\mu)$ be a pre-measure. From $\HH(P)$ define the outer measure $\mu^*$ as done previously. Let $\MM(\mu^*)$ denote the $\sigma$-ring of $\mu^*$-measurable sets contained in $\HH(P).$ then $P\se\MM(\mu^*).$ \\ \\
\pf{
	Let $E,F\in P,$ so that there are disjoint sets $G_1,\hdots,G_n\in P$ such that $E\setminus F=\bigcup_{j=1}^n G_j.$ 
	Then $\mu^*(E\setminus F)\leq\sum_{j=1}^n\mu(G_j).$ By measurability, 
	$$\mu^*(E)=\mu^*(E\cap F)+\sum_{j=1}^n\mu^*(G_j)\geq\mu^*(E\cap F)+\mu^*(E\setminus F).$$
	Therefore $\mu(E)\geq\mu(E\cap F)+\mu^*(E\setminus F).$ Now we must show that, given $A\in\HH(P)$ and $E\in P,$ then $\mu^*(A)\geq\mu^*(A\cap E)+\mu^*(A\setminus E).$ \\
	Let $\epsilon>0.$ There exists $\{F_j\}_{j=1}^{\infty}\se P$ such that $A\se\bigcup_{j=1}^{\infty} F_j,$ and $\mu^*(A)+\epsilon>\sum_{j=1}^{\infty}\mu(F_j).$ Notice that $A\cap E\se\bigcup_{j=1}^{\infty}(F_j\cap E),$ and $A\setminus E\se\bigcup_{j=1}^{\infty}(F_j\setminus E).$ Therefore
	$$\mu^*(A)+\epsilon\geq\sum_{j=1}^{\infty}\mu(F_j)=\sum_{j=1}^{\infty}\Big(\mu^*(F_j\cap E)+\mu^*(F_j\setminus E)\Big)\geq\mu^*(A\cap E)+\mu^*(A\setminus E).$$
	It follows that $\mu^*(A)\geq\mu^*(A\cap E)+\mu^*(A\setminus E),$ so $E$ is $\mu^*$-measurable.
}
\end{prop}

\begin{prop}
Let $(H,\nu)$ be any outer measure. If $A\in\HH,$ and $\nu(A)=0,$ then $A\in\MM(\nu).$ \\ \\
\pf{Let $B\in\HH.$ Then $\nu(B\cap A)=0,$ since $B\cap A\se A,$ and also $\nu(B\cap A^c)\leq\nu(B),$ since $B\cap A^c\se B.$ Measurability of $A$ follows immediately.}
\end{prop}

\noindent In other words, sets of zero outer measure are measurable. 

\begin{defn}
A measure $(\Sm,\mu)$ is \ub{complete} if whenever $E\se\Sm,$ and $\mu(E)=0,$ then for all $A\se E,$ we have $A\in\Sm$ and $\mu(A)=0.$ \\
Given a complete measure, sets whose measure is zero are called \ub{null sets}.
\end{defn}

\begin{itemize}
\item For a complete measure, the null sets form a hereditary $\sigma$-ring.
\item For any outer measure $(H,\nu)$ the corresponding measure on $\MM(\nu)$ is complete (from the last proposition).
\item For a pre-measure $(P,\mu),$ $\mu^*$ on $\Sm(P)$ may not be complete. However, $\Sm(P)\se\MM(\mu^*),$ where $\MM(\mu^*)$ is complete.
\item It's easy to see that for $P=\{[a,b): a<b\},$ that $\Sm(P)$ is exactly the Borel $\sigma$-algebra on $\R{}$ (countable operations on open intervals can yield these half-open intervals).
\item Extensions of measures are in general not unique. 
\end{itemize}

\begin{defn}
Let $\FF$ be a family of subsets of $X,$ and let $\nu:\FF\rightarrow[0,\infty].$ Then $A\in\FF$ is \ub{$\sigma$-finite} if there is $\{B_j\}_{j=1}^{\infty}\se\FF$ such that $A\se\bigcup_{j=1}^{\infty} B_j,$ and $\nu(B_j)<\infty$ for all $j.$
\end{defn}

\noindent We say that $(\FF,\nu)$ is $\sigma$-finite if every $A\in\FF$ is $\sigma$-finite. If $X\in\FF$ and $X$ is $\sigma$-finite, then we say that $(\FF,\nu)$ is totally $\sigma$-finite.

\begin{lemma}
Let $(P,\mu)$ be a pre-measure. Let $\Sm$ be any $\sigma$-ring with $P\se\Sm\se\MM(\mu^*).$ Let $\nu:\Sm\rightarrow[0,\infty]$ and let $\nu$ be countably subadditive, monotone, and $\nu|_P=\mu.$ Then $\nu(A)\leq\mu^*(A)$ for all $A\in\Sm$ such that $A\in\HH(P).$ \\ \\
\pf{For any $\{E_j\}_{j=1}^{\infty}\se P$ such that $A\se\bigcup_{j=1}^{\infty} E_j,$ we have 
$$\nu(A)\leq\sum_{j=1}^{\infty}\nu(E_j)=\sum_{j=1}^{\infty}\mu(E_j).$$
Since $\mu^*(A)=\inf\Big\{\sum_{j=1}^{\infty}\mu(F_j):\;A\se\bigcup_{j=1}^{\infty}F_j\Big\},$ it follows that $\nu(A)\leq\mu^*(A).$}
\end{lemma}

\begin{thm}
Let $(P,\mu)$ be a $\sigma$-finite pre-measure. Then for any $\sigma$-ring $\Sm$ such that $P\se\Sm\se\MM(\mu^*),$ and any measure $\nu$ on $\Sm$ such that $\nu|_P=\mu,$ we have $\nu=\mu^*|_{\Sm}.$ \\ \\
\pf{Let $A\in\Sm.$
\begin{itemize}
\item \underline{Step 1}: Suppose $A\se E,$ $E\in P,$ $\mu(E)<\infty.$ Then $E=A\oplus(E\setminus A),$ so by measurability of $A,$ we have $\nu(E)=\nu(A)+\nu(E\setminus A)\leq\mu^*(A)+\mu^*(E\setminus A)=\mu^*(E)=\mu(E)=\nu(E).$ \\
Thus $\nu(A)+\nu(E\setminus A)=\mu^*(A)+\mu^*(E\setminus A).$ But $\nu(A)\leq\mu^*(A),$ and $\nu(E\setminus A)\leq\mu^*(E\setminus A),$ and since $\mu^*(A)\leq\mu^*(E)<\infty,$ and $\mu^*(E\setminus A)\leq\mu^*(E)<\infty,$ it follows that in fact, $\nu(A)=\mu^*(A),$ and $\nu(E\setminus A)=\mu^*(E\setminus A).$
\item \underline{Step 2}: For general $A\in\Sm,$ there is $\{E_j\}\se P,$ $A\se\bigcup_{j=1}^{\infty}E_j,$ and $\mu(E_j)<\infty$ for all $j.$ Without loss of generality, we can let $\{E_j\}$ be disjoint. Then $A=\bigcup_{j=1}^{\infty}(A\cap E_j),$ so $\nu(A)=\sum_{j=1}^{\infty}\nu(A\cap E_j),$ but $A\cap E_j\se E_j,$ so by step 1, $\nu(A\cap E_j)=\mu^*(A\cap E_j),$ implying that $\mu^*(A)=\sum_{j=1}^{\infty} \mu^*(A\cap E_j).$
\end{itemize}
}
\end{thm}

\begin{defn}
On $\R{},$ define $\alpha(x)=x,$ and consider the pre-measure $\mu_{\alpha}$ defined by $\mu_{\alpha}([a,b))=b-a,$ which can be extended to $\MM(\mu_{\alpha}^*).$ Then $\mu_{\alpha}$ restricted to $\MM(\mu_{\alpha}^*)$ is \ub{Lebesgue measure} on $\R{}.$
\end{defn}

\begin{itemize}
\item Lebesgue measure is translation invariant, i.e., if $A\in\MM(\mu_{\alpha}^*),$ and $r_0\in\R{},$ then if $r_0+A=\{r_0+a:a\in A\},$ we have $\mu_{\alpha}(r_0+A)=\mu_{\alpha}(A).$
\item Above, we can let $\alpha$ be any non-decreasing, left-continuous function. Then the resulting measure is the general \ub{Lebesgue-Stieltjes} measure.
\end{itemize}

\begin{prop}
(Continuity from below): Let $(\Sm,\mu)$ be a measure. If $\{E_j\}\se\Sm,$ and $E_j\se E_{j+1}$ for all $j,$ and if $E=\bigcup_{j=1}^{\infty} E_j,$ then $\mu(E_j)\rightarrow\mu(E).$ \\ \\
\pf{Observe that $E=E_1\oplus\Big(\bigoplus_{j=1}^{\infty}E_{j+1}\setminus E_j\Big),$ so \\
$\mu(E)=\mu(E_1)+\sum_{j=1}^{\infty}\mu(E_{j+1}\setminus E_j)=\lim\limits_{m\rightarrow\infty}\mu(E_{m+1}).$}
\end{prop}

\noindent Observe that if $E_j$ is a decreasing sequence of subsets, i.e., $E_{j+1}\se E_j$ for all $j,$ and if $E=\bigcap_{j=1}^{\infty} E_j,$ then the analogue of the above proposition for intersections is not necessarily true. Consider $E_j=[j,\infty).$ However, we can say the following:

\begin{prop}
(Continuity from above): If $\{E_j\}\se\Sm,$ and $E_{j+1}\se E_j$ for all $j,$ with at least one $\mu(E_j)<\infty,$ and if $E=\bigcap_{j=1}^{\infty} E_j,$ then $\mu(E_j)\rightarrow \mu(E).$ \\ \\
\pf{Note that $\mu(E)<\infty,$ and furthermore, suppose $\mu(E_N)<\infty.$ Then for all $n\geq N,$ $\mu(E_n)<\infty.$ Then the proof follows in a similar manner as above.}
\end{prop}

\section{Non-Measurable Sets}
The construction of non-measurable sets often makes use of the axiom of choice. The following is a well-known example. \\
\Ex Let $T=\{z\in\C{}:\;|z|=1\}.$ Let $H=\{z\in T:\exists n\in\mathbb{Z}\text{ such that } z^n=1\}.$ Then $T$ is a compact group under multiplication, and $H$ is a subgroup, so we can form the cosets of $H.$ \\
Let $A$ be a set that contains exactly one element from each coset of $H.$ By rotation invariance, $\mu(hA)=\mu(A)$ for any $h\in H.$ Also, since $H$ is countable, $T=\bigoplus_{h\in H} hA,$ so $\mu(T)=\sum_{h\in H}\mu(hA)=\sum_{h\in H}\mu(A).$ \\
Clearly, $0<\mu(T)<\infty.$ If $\mu(A)=0,$ then this would yield $\mu(T)=0,$ a contradiction. But if $\mu(A)>0,$ then $\mu(T)=\infty,$ a contradiction. Therefore $A$ is not measurable.